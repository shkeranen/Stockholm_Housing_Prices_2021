% Options for packages loaded elsewhere
\PassOptionsToPackage{unicode}{hyperref}
\PassOptionsToPackage{hyphens}{url}
%
\documentclass[
]{article}
\title{Stockholm Housing Prices}
\author{Selja}
\date{1/9/2022}

\usepackage{amsmath,amssymb}
\usepackage{lmodern}
\usepackage{iftex}
\ifPDFTeX
  \usepackage[T1]{fontenc}
  \usepackage[utf8]{inputenc}
  \usepackage{textcomp} % provide euro and other symbols
\else % if luatex or xetex
  \usepackage{unicode-math}
  \defaultfontfeatures{Scale=MatchLowercase}
  \defaultfontfeatures[\rmfamily]{Ligatures=TeX,Scale=1}
\fi
% Use upquote if available, for straight quotes in verbatim environments
\IfFileExists{upquote.sty}{\usepackage{upquote}}{}
\IfFileExists{microtype.sty}{% use microtype if available
  \usepackage[]{microtype}
  \UseMicrotypeSet[protrusion]{basicmath} % disable protrusion for tt fonts
}{}
\makeatletter
\@ifundefined{KOMAClassName}{% if non-KOMA class
  \IfFileExists{parskip.sty}{%
    \usepackage{parskip}
  }{% else
    \setlength{\parindent}{0pt}
    \setlength{\parskip}{6pt plus 2pt minus 1pt}}
}{% if KOMA class
  \KOMAoptions{parskip=half}}
\makeatother
\usepackage{xcolor}
\IfFileExists{xurl.sty}{\usepackage{xurl}}{} % add URL line breaks if available
\IfFileExists{bookmark.sty}{\usepackage{bookmark}}{\usepackage{hyperref}}
\hypersetup{
  pdftitle={Stockholm Housing Prices},
  pdfauthor={Selja},
  hidelinks,
  pdfcreator={LaTeX via pandoc}}
\urlstyle{same} % disable monospaced font for URLs
\usepackage[margin=1in]{geometry}
\usepackage{color}
\usepackage{fancyvrb}
\newcommand{\VerbBar}{|}
\newcommand{\VERB}{\Verb[commandchars=\\\{\}]}
\DefineVerbatimEnvironment{Highlighting}{Verbatim}{commandchars=\\\{\}}
% Add ',fontsize=\small' for more characters per line
\usepackage{framed}
\definecolor{shadecolor}{RGB}{248,248,248}
\newenvironment{Shaded}{\begin{snugshade}}{\end{snugshade}}
\newcommand{\AlertTok}[1]{\textcolor[rgb]{0.94,0.16,0.16}{#1}}
\newcommand{\AnnotationTok}[1]{\textcolor[rgb]{0.56,0.35,0.01}{\textbf{\textit{#1}}}}
\newcommand{\AttributeTok}[1]{\textcolor[rgb]{0.77,0.63,0.00}{#1}}
\newcommand{\BaseNTok}[1]{\textcolor[rgb]{0.00,0.00,0.81}{#1}}
\newcommand{\BuiltInTok}[1]{#1}
\newcommand{\CharTok}[1]{\textcolor[rgb]{0.31,0.60,0.02}{#1}}
\newcommand{\CommentTok}[1]{\textcolor[rgb]{0.56,0.35,0.01}{\textit{#1}}}
\newcommand{\CommentVarTok}[1]{\textcolor[rgb]{0.56,0.35,0.01}{\textbf{\textit{#1}}}}
\newcommand{\ConstantTok}[1]{\textcolor[rgb]{0.00,0.00,0.00}{#1}}
\newcommand{\ControlFlowTok}[1]{\textcolor[rgb]{0.13,0.29,0.53}{\textbf{#1}}}
\newcommand{\DataTypeTok}[1]{\textcolor[rgb]{0.13,0.29,0.53}{#1}}
\newcommand{\DecValTok}[1]{\textcolor[rgb]{0.00,0.00,0.81}{#1}}
\newcommand{\DocumentationTok}[1]{\textcolor[rgb]{0.56,0.35,0.01}{\textbf{\textit{#1}}}}
\newcommand{\ErrorTok}[1]{\textcolor[rgb]{0.64,0.00,0.00}{\textbf{#1}}}
\newcommand{\ExtensionTok}[1]{#1}
\newcommand{\FloatTok}[1]{\textcolor[rgb]{0.00,0.00,0.81}{#1}}
\newcommand{\FunctionTok}[1]{\textcolor[rgb]{0.00,0.00,0.00}{#1}}
\newcommand{\ImportTok}[1]{#1}
\newcommand{\InformationTok}[1]{\textcolor[rgb]{0.56,0.35,0.01}{\textbf{\textit{#1}}}}
\newcommand{\KeywordTok}[1]{\textcolor[rgb]{0.13,0.29,0.53}{\textbf{#1}}}
\newcommand{\NormalTok}[1]{#1}
\newcommand{\OperatorTok}[1]{\textcolor[rgb]{0.81,0.36,0.00}{\textbf{#1}}}
\newcommand{\OtherTok}[1]{\textcolor[rgb]{0.56,0.35,0.01}{#1}}
\newcommand{\PreprocessorTok}[1]{\textcolor[rgb]{0.56,0.35,0.01}{\textit{#1}}}
\newcommand{\RegionMarkerTok}[1]{#1}
\newcommand{\SpecialCharTok}[1]{\textcolor[rgb]{0.00,0.00,0.00}{#1}}
\newcommand{\SpecialStringTok}[1]{\textcolor[rgb]{0.31,0.60,0.02}{#1}}
\newcommand{\StringTok}[1]{\textcolor[rgb]{0.31,0.60,0.02}{#1}}
\newcommand{\VariableTok}[1]{\textcolor[rgb]{0.00,0.00,0.00}{#1}}
\newcommand{\VerbatimStringTok}[1]{\textcolor[rgb]{0.31,0.60,0.02}{#1}}
\newcommand{\WarningTok}[1]{\textcolor[rgb]{0.56,0.35,0.01}{\textbf{\textit{#1}}}}
\usepackage{graphicx}
\makeatletter
\def\maxwidth{\ifdim\Gin@nat@width>\linewidth\linewidth\else\Gin@nat@width\fi}
\def\maxheight{\ifdim\Gin@nat@height>\textheight\textheight\else\Gin@nat@height\fi}
\makeatother
% Scale images if necessary, so that they will not overflow the page
% margins by default, and it is still possible to overwrite the defaults
% using explicit options in \includegraphics[width, height, ...]{}
\setkeys{Gin}{width=\maxwidth,height=\maxheight,keepaspectratio}
% Set default figure placement to htbp
\makeatletter
\def\fps@figure{htbp}
\makeatother
\setlength{\emergencystretch}{3em} % prevent overfull lines
\providecommand{\tightlist}{%
  \setlength{\itemsep}{0pt}\setlength{\parskip}{0pt}}
\setcounter{secnumdepth}{-\maxdimen} % remove section numbering
\ifLuaTeX
  \usepackage{selnolig}  % disable illegal ligatures
\fi

\begin{document}
\maketitle

\hypertarget{setting-up-my-enviroment}{%
\subsection{Setting up my enviroment}\label{setting-up-my-enviroment}}

\begin{Shaded}
\begin{Highlighting}[]
\FunctionTok{library}\NormalTok{(sf)}
\end{Highlighting}
\end{Shaded}

\begin{verbatim}
## Linking to GEOS 3.9.1, GDAL 3.2.1, PROJ 7.2.1; sf_use_s2() is TRUE
\end{verbatim}

\begin{Shaded}
\begin{Highlighting}[]
\FunctionTok{library}\NormalTok{(tidyverse)}
\end{Highlighting}
\end{Shaded}

\begin{verbatim}
## -- Attaching packages --------------------------------------- tidyverse 1.3.1 --
\end{verbatim}

\begin{verbatim}
## v ggplot2 3.3.5     v purrr   0.3.4
## v tibble  3.1.6     v dplyr   1.0.7
## v tidyr   1.1.4     v stringr 1.4.0
## v readr   2.1.1     v forcats 0.5.1
\end{verbatim}

\begin{verbatim}
## -- Conflicts ------------------------------------------ tidyverse_conflicts() --
## x dplyr::filter() masks stats::filter()
## x dplyr::lag()    masks stats::lag()
\end{verbatim}

\begin{Shaded}
\begin{Highlighting}[]
\FunctionTok{library}\NormalTok{(jsonlite)}
\end{Highlighting}
\end{Shaded}

\begin{verbatim}
## 
## Attaching package: 'jsonlite'
\end{verbatim}

\begin{verbatim}
## The following object is masked from 'package:purrr':
## 
##     flatten
\end{verbatim}

\begin{Shaded}
\begin{Highlighting}[]
\FunctionTok{library}\NormalTok{(}\StringTok{"dplyr"}\NormalTok{) }
\FunctionTok{library}\NormalTok{(}\StringTok{"tidyr"}\NormalTok{)}
\end{Highlighting}
\end{Shaded}

\hypertarget{accesing-the-geo-location-data-for-sweden}{%
\subsection{Accesing the GEO location data for
Sweden}\label{accesing-the-geo-location-data-for-sweden}}

\begin{Shaded}
\begin{Highlighting}[]
\NormalTok{tmp }\OtherTok{\textless{}{-}} \FunctionTok{tempfile}\NormalTok{()}
\FunctionTok{download.file}\NormalTok{(}\StringTok{"http://api.thenmap.net/v2/se{-}7/geo/2020{-}06{-}06"}\NormalTok{, }\AttributeTok{destfile =}\NormalTok{ tmp)}

\NormalTok{mun\_name }\OtherTok{\textless{}{-}} \FunctionTok{fromJSON}\NormalTok{(}\StringTok{"http://api.thenmap.net/v2/se{-}7/data/2020{-}06{-}06?language=sv\&data\_props=name|shapeid|is\_in"}\NormalTok{) }\SpecialCharTok{\%\textgreater{}\%} 
  \FunctionTok{unnest}\NormalTok{(is\_in) }\SpecialCharTok{\%\textgreater{}\%} 
  \FunctionTok{rename}\NormalTok{(}\AttributeTok{county =}\NormalTok{ is\_in)}
\end{Highlighting}
\end{Shaded}

\hypertarget{joining-the-two-datasets}{%
\subsection{Joining the two datasets}\label{joining-the-two-datasets}}

\begin{Shaded}
\begin{Highlighting}[]
\NormalTok{mun }\OtherTok{\textless{}{-}} \FunctionTok{read\_sf}\NormalTok{(tmp) }\SpecialCharTok{\%\textgreater{}\%} 
  \FunctionTok{left\_join}\NormalTok{(mun\_name, }\AttributeTok{by =} \FunctionTok{c}\NormalTok{(}\StringTok{"id"} \OtherTok{=} \StringTok{"shapeid"}\NormalTok{)) }
\end{Highlighting}
\end{Shaded}

\hypertarget{filtering-the-dataset-to-show-only-stockholm-area}{%
\subsection{Filtering the dataset to show only Stockholm
area}\label{filtering-the-dataset-to-show-only-stockholm-area}}

\begin{Shaded}
\begin{Highlighting}[]
\NormalTok{stockholm\_mun }\OtherTok{\textless{}{-}}\NormalTok{ mun }\SpecialCharTok{\%\textgreater{}\%} 
  \FunctionTok{filter}\NormalTok{(county }\SpecialCharTok{==} \StringTok{"Stockholms län"}\NormalTok{) }\SpecialCharTok{\%\textgreater{}\%} 
  \FunctionTok{mutate}\NormalTok{(}\AttributeTok{fill\_data =} \FunctionTok{rnorm}\NormalTok{(}\FunctionTok{nrow}\NormalTok{(.))) }
\end{Highlighting}
\end{Shaded}

\#Plotting the dataset

\begin{Shaded}
\begin{Highlighting}[]
\FunctionTok{ggplot}\NormalTok{(stockholm\_mun) }\SpecialCharTok{+}
  \FunctionTok{geom\_sf}\NormalTok{(}\FunctionTok{aes}\NormalTok{(}\AttributeTok{fill =}\NormalTok{ fill\_data)) }\SpecialCharTok{+}
  \FunctionTok{scale\_fill\_viridis\_c}\NormalTok{() }\SpecialCharTok{+}
  \FunctionTok{theme\_void}\NormalTok{()}
\end{Highlighting}
\end{Shaded}

\includegraphics{Housing_Prices_Stockholm_files/figure-latex/unnamed-chunk-2-1.pdf}
\#\# Let's add the average housing prices. This data can be find at
\url{https://www.maklarstatistik.se/} and check the column names.

\begin{Shaded}
\begin{Highlighting}[]
\FunctionTok{library}\NormalTok{(readxl)}

\NormalTok{house\_price }\OtherTok{\textless{}{-}} \FunctionTok{read\_excel}\NormalTok{(}\StringTok{"housing\_stockholm\_2021.xlsx"}\NormalTok{)}

\FunctionTok{colnames}\NormalTok{(house\_price)}
\end{Highlighting}
\end{Shaded}

\begin{verbatim}
## [1] "Områden"            "Antal sålda"        "Kr/kvm"            
## [4] "Medelpris kr"       "Prisutveckling (%)" "55 m2"
\end{verbatim}

\hypertarget{renaming-some-of-the-column-names-for-merging-and-clarity}{%
\subsection{Renaming some of the column names for merging and
clarity}\label{renaming-some-of-the-column-names-for-merging-and-clarity}}

\begin{Shaded}
\begin{Highlighting}[]
\FunctionTok{names}\NormalTok{(house\_price) [}\FunctionTok{names}\NormalTok{(house\_price) }\SpecialCharTok{==} \StringTok{"Områden"}\NormalTok{] }\OtherTok{\textless{}{-}} \StringTok{"name"}
\FunctionTok{names}\NormalTok{(house\_price) [}\FunctionTok{names}\NormalTok{(house\_price) }\SpecialCharTok{==} \StringTok{"Kr/kvm"}\NormalTok{] }\OtherTok{\textless{}{-}} \StringTok{"PricePerSqm"}

\FunctionTok{View}\NormalTok{(house\_price)}
\end{Highlighting}
\end{Shaded}

\hypertarget{merging-housing-prices-and-geo-data-by-name-column}{%
\subsection{Merging housing prices and geo data by name
column}\label{merging-housing-prices-and-geo-data-by-name-column}}

\begin{Shaded}
\begin{Highlighting}[]
\NormalTok{total }\OtherTok{\textless{}{-}} \FunctionTok{merge}\NormalTok{(stockholm\_mun,house\_price, }\AttributeTok{by =} \StringTok{"name"}\NormalTok{)}
\end{Highlighting}
\end{Shaded}

\hypertarget{plotting-the-data.}{%
\subsection{Plotting the data.}\label{plotting-the-data.}}

\begin{Shaded}
\begin{Highlighting}[]
\FunctionTok{ggplot}\NormalTok{(}\AttributeTok{data =}\NormalTok{ total) }\SpecialCharTok{+}
  \FunctionTok{geom\_sf}\NormalTok{() }\SpecialCharTok{+}
  \FunctionTok{geom\_sf}\NormalTok{(}\AttributeTok{data =}\NormalTok{ total, }\FunctionTok{aes}\NormalTok{(}\AttributeTok{fill =}\NormalTok{ PricePerSqm)) }\SpecialCharTok{+}
  \FunctionTok{scale\_fill\_viridis\_c}\NormalTok{(}\AttributeTok{trans =} \StringTok{"sqrt"}\NormalTok{, }\AttributeTok{alpha =}\NormalTok{ .}\DecValTok{4}\NormalTok{) }\SpecialCharTok{+}
  \FunctionTok{theme\_void}\NormalTok{()}
\end{Highlighting}
\end{Shaded}

\includegraphics{Housing_Prices_Stockholm_files/figure-latex/unnamed-chunk-6-1.pdf}

\hypertarget{lets-break-the-price-per-sqm-to-intervals}{%
\subsection{Let's break the Price per Sqm to
intervals}\label{lets-break-the-price-per-sqm-to-intervals}}

\begin{Shaded}
\begin{Highlighting}[]
\FunctionTok{library}\NormalTok{(classInt)}

\NormalTok{breaks\_qt }\OtherTok{\textless{}{-}} \FunctionTok{classIntervals}\NormalTok{(}\FunctionTok{c}\NormalTok{(}\FunctionTok{min}\NormalTok{(total}\SpecialCharTok{$}\NormalTok{PricePerSqm),}\DecValTok{0}\NormalTok{,total}\SpecialCharTok{$}\NormalTok{PricePerSqm), }\AttributeTok{n =} \DecValTok{7}\NormalTok{, }\AttributeTok{style =} \StringTok{"quantile"}\NormalTok{)}

\NormalTok{breaks\_qt}
\end{Highlighting}
\end{Shaded}

\begin{verbatim}
## style: quantile
##   one of 177,100 possible partitions of this variable into 7 classes
##        [0,30386.43) [30386.43,33968.57) [33968.57,36157.71) [36157.71,43007.71) 
##                   4                   4                   4                   3 
## [43007.71,48984.43) [48984.43,61014.57)    [61014.57,81393] 
##                   4                   4                   4
\end{verbatim}

\begin{Shaded}
\begin{Highlighting}[]
\NormalTok{total }\OtherTok{\textless{}{-}} \FunctionTok{mutate}\NormalTok{(total, }\AttributeTok{PricePerSqm\_cat =} \FunctionTok{cut}\NormalTok{(PricePerSqm, breaks\_qt}\SpecialCharTok{$}\NormalTok{brks)) }

\NormalTok{total }\OtherTok{\textless{}{-}} \FunctionTok{cbind}\NormalTok{(total, }\FunctionTok{st\_coordinates}\NormalTok{(}\FunctionTok{st\_centroid}\NormalTok{(total)))}
\end{Highlighting}
\end{Shaded}

\begin{verbatim}
## Warning in st_centroid.sf(total): st_centroid assumes attributes are constant
## over geometries of x
\end{verbatim}

\hypertarget{plotting-the-data-with-new-intervals}{%
\subsection{Plotting the data with new
intervals}\label{plotting-the-data-with-new-intervals}}

\begin{Shaded}
\begin{Highlighting}[]
\FunctionTok{ggplot}\NormalTok{(total) }\SpecialCharTok{+} 
  \FunctionTok{geom\_sf}\NormalTok{(}\FunctionTok{aes}\NormalTok{(}\AttributeTok{fill=}\NormalTok{PricePerSqm\_cat)) }\SpecialCharTok{+}
  \FunctionTok{geom\_label}\NormalTok{(}\AttributeTok{data =}\NormalTok{ total, }\FunctionTok{aes}\NormalTok{(X, Y, }\AttributeTok{label =}\NormalTok{ name), }\AttributeTok{size =} \DecValTok{3}\NormalTok{) }\SpecialCharTok{+}
  \FunctionTok{scale\_fill\_brewer}\NormalTok{(}\AttributeTok{palette =} \StringTok{"OrRd"}\NormalTok{) }
\end{Highlighting}
\end{Shaded}

\includegraphics{Housing_Prices_Stockholm_files/figure-latex/unnamed-chunk-8-1.pdf}

\hypertarget{to-avoid-the-overlapping-ggrepel-library-is-applied}{%
\subsection{To avoid the overlapping, ggrepel library is
applied}\label{to-avoid-the-overlapping-ggrepel-library-is-applied}}

\begin{Shaded}
\begin{Highlighting}[]
\FunctionTok{library}\NormalTok{(ggrepel)}
\end{Highlighting}
\end{Shaded}

\hypertarget{lets-plot-the-result-to-see-what-is-the-outcome-now.}{%
\subsection{Let's plot the result to see what is the outcome
now.}\label{lets-plot-the-result-to-see-what-is-the-outcome-now.}}

\begin{Shaded}
\begin{Highlighting}[]
\FunctionTok{ggplot}\NormalTok{(total) }\SpecialCharTok{+} 
  \FunctionTok{geom\_sf}\NormalTok{(}\FunctionTok{aes}\NormalTok{(}\AttributeTok{fill=}\NormalTok{PricePerSqm\_cat)) }\SpecialCharTok{+}
  \FunctionTok{geom\_text\_repel}\NormalTok{(}\AttributeTok{data =}\NormalTok{ total, }\FunctionTok{aes}\NormalTok{(X, Y, }\AttributeTok{label =}\NormalTok{ name))}\SpecialCharTok{+}
  \FunctionTok{scale\_fill\_brewer}\NormalTok{(}\AttributeTok{palette =} \StringTok{"OrRd"}\NormalTok{) }
\end{Highlighting}
\end{Shaded}

\begin{verbatim}
## Warning: ggrepel: 9 unlabeled data points (too many overlaps). Consider
## increasing max.overlaps
\end{verbatim}

\includegraphics{Housing_Prices_Stockholm_files/figure-latex/unnamed-chunk-10-1.pdf}

\hypertarget{still-quite-hard-to-read-lets-try-without-labels.}{%
\subsection{Still quite hard to read, let's try without
labels.}\label{still-quite-hard-to-read-lets-try-without-labels.}}

\begin{Shaded}
\begin{Highlighting}[]
\FunctionTok{ggplot}\NormalTok{(total) }\SpecialCharTok{+} 
  \FunctionTok{geom\_sf}\NormalTok{(}\FunctionTok{aes}\NormalTok{(}\AttributeTok{fill =}\NormalTok{ PricePerSqm\_cat), }\AttributeTok{show.legend =} \ConstantTok{TRUE}\NormalTok{) }\SpecialCharTok{+} 
  \FunctionTok{geom\_sf\_label}\NormalTok{(}\FunctionTok{aes}\NormalTok{(}\AttributeTok{label =} \ConstantTok{NA}\NormalTok{), }\AttributeTok{label.padding =} \FunctionTok{unit}\NormalTok{(}\DecValTok{1}\NormalTok{, }\StringTok{"mm"}\NormalTok{)) }\SpecialCharTok{+}
  \FunctionTok{scale\_fill\_brewer}\NormalTok{(}\AttributeTok{palette =} \StringTok{"OrRd"}\NormalTok{)}
\end{Highlighting}
\end{Shaded}

\begin{verbatim}
## Warning in st_point_on_surface.sfc(sf::st_zm(x)): st_point_on_surface may not
## give correct results for longitude/latitude data
\end{verbatim}

\begin{verbatim}
## Warning: Removed 25 rows containing missing values (geom_label).
\end{verbatim}

\includegraphics{Housing_Prices_Stockholm_files/figure-latex/unnamed-chunk-11-1.pdf}

\hypertarget{lets-try-with-another-color-scheme-and-some-of-the-labels.}{%
\subsection{Let's try with another color scheme and some of the
labels.}\label{lets-try-with-another-color-scheme-and-some-of-the-labels.}}

\begin{Shaded}
\begin{Highlighting}[]
\FunctionTok{ggplot}\NormalTok{(total) }\SpecialCharTok{+} 
  \FunctionTok{geom\_sf}\NormalTok{(}\FunctionTok{aes}\NormalTok{(}\AttributeTok{fill=}\NormalTok{PricePerSqm\_cat)) }\SpecialCharTok{+}
  \FunctionTok{geom\_text\_repel}\NormalTok{(}\AttributeTok{data =}\NormalTok{ total, }\FunctionTok{aes}\NormalTok{(X, Y, }\AttributeTok{label =}\NormalTok{ name), }\AttributeTok{nudge\_y =} \FunctionTok{c}\NormalTok{(}\FloatTok{0.04}\NormalTok{))}\SpecialCharTok{+}
  \FunctionTok{scale\_fill\_viridis\_d}\NormalTok{()}
\end{Highlighting}
\end{Shaded}

\begin{verbatim}
## Warning: ggrepel: 11 unlabeled data points (too many overlaps). Consider
## increasing max.overlaps
\end{verbatim}

\includegraphics{Housing_Prices_Stockholm_files/figure-latex/unnamed-chunk-12-1.pdf}

\hypertarget{for-future-work-it-would-be-handy-to-cut-the-amount-of-labels-for-readability}{%
\subsubsection{For future work, it would be handy to cut the amount of
labels for
readability}\label{for-future-work-it-would-be-handy-to-cut-the-amount-of-labels-for-readability}}

\hypertarget{interval-can-be-changed-to-enchance-the-readability}{%
\subsubsection{Interval can be changed to enchance the
readability}\label{interval-can-be-changed-to-enchance-the-readability}}

\hypertarget{naming-and-other-visual-components-should-be-added}{%
\subsubsection{Naming and other visual components should be
added}\label{naming-and-other-visual-components-should-be-added}}

\hypertarget{but-this-will-do-for-now-such-a-fun-saturday-project}{%
\subsubsection{But this will do for now, such a fun Saturday
project!}\label{but-this-will-do-for-now-such-a-fun-saturday-project}}

\end{document}
